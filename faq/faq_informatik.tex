\begin{Huge}
    Informatik-FAQ
\end{Huge}

\begin{block}{Häufig gestellte Fragen zum Studium}

\question{Lernt man im Studium, wie man programmiert?}{
    Ja, aber auf einer sehr eigenständigen Basis. Man entwickelt ein Verständnis für den Aufbau und die Funktionsweise von Programmiersprachen. Alles andere, wasa darüber hinaus geht, muss man sich selbst aneignen.    
    %Man bekommt eine Überblick über die Sprache(n), alles andere was darüber hinaus geht muss man sich selbst aneignen.
}

\question{Welche Programmiersprachen macht man da so?}{
    Ist vom Professor abhängig. In den ersten beiden Semestern meistens entweder Java oder Racket, manchmal auch C++.
}

\question{Muss man programmieren können, um das Studium anzufangen?}{
    Nein. Die Vorlesung beginnt absolut bei 0, um allen den Einstieg zu ermöglichen.
}

\question{Muss man gut in Mathe sein?}{
    Man muss kein Mathe-Genie sein, sollte Mathe aber auch nicht hassen. Gerade am Anfang des Studium hat man viel Mathe. Dabei entsteht durch die gemeinsame Mathe-Erfahrung aber auch schnell ein Gemeinschaftsgefühl.
}

\question{Brauche ich einen eigenen Laptop?}{
    Ist empfehlenswert. Die Anzahl an Rechnern in den Rechnerräumen ist begrenzt und mit dem eigenen Laptop ist man um einiges flexibler. Tipp: Nicht die Gaming-Maschine, maximal 14 Zoll und lange Akkulaufzeit. Betriebssystem vollkommen egal.
}

\question{Wie ist die Frauenquote so?}{
    Mit 17\% leider weit unter 50\%. Durch den höheren Anteil an Frauen in den Micsch- Informatikfächern merkt man das aber gar nicht so stark. 
}

\question{Ich zocke total gerne, hab ich das Zeug, um Informatik zu studieren?}{
    Informatik ist ungleich Zocken. Du musst analytisches Denken entwickeln, Spaß am Ausprobieren besitzen und keinen Hass auf Mathe haben (den entwickelt man im Studium dann sowieso). Ein Interesse an Videospielen und Technik schadet aber bestimmt nicht. 
}

\question{Was arbeitet man danach so?}{
    Alle Bereiche der IT-Branche, z. B. Softwareentwicklung und -Beratung, Hardware-Entwicklung, Automatisierung, Automobilindustrie, Unternehmensberatung, Handel, Banken, Versicherungen...
}

\question{Gibt es Praktika?}{
    Im normalen Studienverlauf ist kein berufsorientiertes Praktikum vorgesehen, viele arbeiten aber parallel als Werksstudent oder man macht ein Kurzpraktikum in den Semesterferien.
}

\question{Kann man ein Auslandssemster machen?}{
    Klar, geht immer. Tübingen nimmt am ERASMUS-Programm teil, die Organisation ist aber langwierig und man sollte sich früh (ein Jahr vorher) drum kümmern.
}

\question{Wie ist da so der NC?}{
    An der Uni Tübingen gibt es keinen NC für die reine Informatik.
}

\question{Laptop oder Tablet?}{
    Viele Programme die im Studium verwendet werden laufen leider nicht auf Android, IOS und Ähnlichem. Deswegen lautet die Antwort bei einem Entweder-Oder klar Laptop. Wer Geld für beides hat ist mit einem zusätzlichen Tablet oder Laptop mit Touch aber auch gut bedient, da gerade Mathe zu Beginn oft leicher von der Hand geht. 
}

%\question{Wie kann ich mich aufs Studium vorbereiten?}{
%    Die Uni bietet einen Mathe-Vorkurs für Informatiker an, der nicht nur hilft deine Mathe-Kentnisse aufzufrischen %und den Einstieg in die Uni-Mathematik zu finden, sondern auch Möglichkeit bietet andere Studierende kennen zu %lernen. 
%}
\end{block}
