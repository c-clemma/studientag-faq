\colorlet{medienwiss}{andere}
\DTLsetpiesegmentcolor{1}{info}
\DTLsetpiesegmentcolor{2}{mathe}
\DTLsetpiesegmentcolor{3}{medieninfo}
\DTLsetpiesegmentcolor{4}{medienwiss}
\DTLsetpiesegmentcolor{5}{sq}

\begin{Huge}
    Medieninformatik
\end{Huge}

\begin{exampleblock}{\textcolor{white}{Was ist der Studiengang?}}
    Irgendwas mit Medien? Nicht so wirklich. Die Medieninformatik beschäftigt sich mit User Interfaces, Nutzerinteraktion, modernen Techniken wie Eye Tracking, macht Ausflüge in die Medienwissenschaft aber ist auch zu großen Teilen Informatik- und Mathematik-lastig. Danach kann das Studium mit einem Master (4 Semester Regelstudienzeit) weitergeführt werden.
\end{exampleblock}

\begin{block}{Welcher Teil macht wie viel im Studium aus?}
    \begin{figure}[h!]
        \vspace{-10pt}
        \begin{minipage}{\linewidth}
            \centering
            \DTLloaddb{LPverteilungMedien}{inhalte/medieninfo.csv}
            \tikzstyle{every node}=[text width={},minimum height=0pt]
            \DTLpiechart{
                variable=\lp,
                innerlabel={\parbox{40pt}{\centering\color{white} \bereich}},
                innerratio=0.25,
                radius=70pt,
                rotateinner}{LPverteilungMedien}{\bereich=Bereich,\lp=LP}
        \end{minipage}
        \caption{Verteilung der Themenbereiche über das komplette Studium}
    \end{figure}
\end{block}

\begin{block}{Was macht man in welchem Semester?}
    \vspace{-10pt}
    \begin{figure}[h!]
        \resizebox{\linewidth}{!}{
        \begin{minipage}{\textwidth}
            \begin{tikzpicture}
                \begin{scope}[start chain=going below,node distance=\nodedistancecorrection]
                    \semester{1}
                    \veranstaltung{9}{Praktische In\-for\-ma\-tik~1}{info}
                    \veranstaltung{9}{Mathe\-matik f. Informatik~1}{mathe}
                    \veranstaltung{6}{User-Experience}{medieninfo}
                    \veranstaltung{9}{WP Medienwiss.}{medienwiss}
                    \SumLP{33}
                \end{scope}
                \begin{scope}[xshift=1\lpwidth,start chain=going below,node distance=\nodedistancecorrection]
                    \semester{2}
                    \veranstaltung{9}{Praktische In\-for\-ma\-tik~2}{info}
                    \veranstaltung{9}{Technische Informatik~2}{info}
                    \veranstaltung{9}{Mathe\-matik f. Informatik~2}{mathe}
                    \veranstaltung{6}{Einführung in die Internettech.}{medieninfo}
                    \SumLP{33}
                \end{scope}
                \begin{scope}[xshift=2\lpwidth,start chain=going below,node distance=\nodedistancecorrection]
                    \semester{3}
                    \veranstaltung{9}{Theoretische Informatik 1}{info}
                    \veranstaltung{6}{Praktische Informatik 3}{info}
                    \veranstaltung{9}{Mathe\-matik f. Informatik~3}{mathe}
                    \veranstaltung{6}{Grundlagen der Multimediatechnik}{medieninfo}
                    \SumLP{30}
                \end{scope}
                \begin{scope}[xshift=3\lpwidth,start chain=going below,node distance=\nodedistancecorrection]
                    \semester{4}                
                    \veranstaltung{9}{The\-o\-re\-ti\-sche Informatik~2}{info}
                    \veranstaltung{9}{Team\-projekt}{sq}
                    \veranstaltung{6}{WP Informatik}{info}
                    \veranstaltung{3}{Ethik-Proseminar}{sq} 
                    \SumLP{30}
                \end{scope}
                \begin{scope}[xshift=4\lpwidth,start chain=going below,node distance=\nodedistancecorrection]
                    \semester{5}
                    \veranstaltung{6}{Bildverarbeitung}{medieninfo}
                    \veranstaltung{9}{Graphische Datenverarbeitung}{medieninfo}
                    \veranstaltung{6}{WP Medieninformatik}{medieninfo}
                    \veranstaltung{3}{WP Medienwiss.}{medienwiss}
                    \veranstaltung{3}{Proseminar}{sq}
                    \SumLP{27}
                \end{scope}
                \begin{scope}[xshift=5\lpwidth,start chain=going below,node distance=\nodedistancecorrection]
                    \semester{6}
                    \veranstaltung{9}{WP Medieninformatik}{medieninfo}
                    \veranstaltung{6}{übK}{sq}
                    \veranstaltung{15}{Bachelor\-arbeit inkl Vortrag}{medieninfo}
                    \SumLP{27}
                \end{scope}
            \end{tikzpicture}
        \end{minipage}}  
    \end{figure}
    
    Das 1. Semester ist nach Plan ein Wintersemester, der Studienbeginn ist hier auch nur zum Wintersemester möglich. 
    Dieser Verlauf ist lediglich ein Vorschlag und kein bindender Studienplan. Es empfiehlt sich jedoch, den Plan einzuhalten, wenn man in Regelstudienzeit studieren möchte.
\end{block}

\vfill
\begin{flushright}
    \includegraphics[width=0.4\textwidth]{images/fsilogo.pdf}
\end{flushright}
