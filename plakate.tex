\documentclass{beamer}
\usepackage[size=a4]{beamerposter}
\beamertemplatenavigationsymbolsempty
\usecolortheme{parrot}
\usepackage[utf8]{inputenc}
\usepackage{subfig}
\usepackage{color}
\usepackage{hyperref}
\usepackage{url}
\usepackage{graphicx}
\usepackage{enumitem}
\usepackage[scale=\scaleFactor]{geometry}
\setenumerate{label=\alph*)}
\usepackage{ngerman}
\usepackage{eurosym}
\usepackage{pgf,tikz}
\usetikzlibrary{arrows}
\usepackage{mathtools}
\usepackage{afterpage}
\usepackage{mathrsfs}
% This code makes the paper a3 and a portrait
\geometry{a5paper, portrait,top=10mm, right=15mm, bottom=10mm, left=15mm}
\author{Tim Beckmann, Alexander Phi. Goetz}


\usepackage{datatool,datapie}
\usepackage{transparent}

\definecolor{fsiblue}{HTML}{000080}

\setbeamercolor{block body}{bg=fsiblue!10}
\setbeamercolor{block title}{bg=fsiblue}
\usetikzlibrary{chains}

\setlength{\DTLpieoutlinewidth}{1pt}
\renewcommand*{\DTLpieoutlinecolor}{fsiblue!10}
\renewcommand*{\DTLdisplayinnerlabel}[1]{\textsf{#1}}
% \renewcommand*{\DTLdisplayouterlabel}[1]{%
% \DTLdocurrentpiesegmentcolor
% \textsf{\shortstack{#1}}}

\pdfpageattr {/Group << /S /Transparency /I true /CS /DeviceRGB>>}

\colorlet{medieninfo}{lightblue}
\colorlet{info}{blue}
\colorlet{bioinfo}{lightgreen}
\colorlet{medizininfo}{green}
\colorlet{kogni}{yellow}

\colorlet{mathe}{darkblue!80}%{turquois}
\colorlet{andere}{brown}
\colorlet{sq}{grayviolet}

\colorlet{neutralbg}{black!30}

\newlength{\lpheight}
\setlength{\lpheight}{4pt}
\newlength{\lpwidth}
\setlength{\lpwidth}{55pt}
\newlength{\nodedistancecorrection}
\setlength{\nodedistancecorrection}{-1pt}

\newcommand{\scaleFactor}{2} % Skalierungsfaktor von A5 auf A3

\newcommand{\veranstaltung}[3]{% #1: Umfang, #2: Titel, #3: Kategorie/Farbe
  \node[anchor=north,draw=fsiblue!10,line width=1pt,inner ysep=0pt,inner xsep=0cm,text width=\lpwidth,fill=#3,minimum height=#1\lpheight,on chain,text badly centered,text=white]{#2};}

\newcommand{\semester}[1]{% #1: Semesterzahl
  \node[anchor=north,draw=white,fill=neutralbg,line width=2pt,inner ysep=4pt,inner xsep=0pt,text width=\lpwidth,minimum height=1.5\lpheight,on chain,text badly centered]{#1.\ Semester};}

\newcommand{\SumLP}[1]{% #1: Summe LP pro Semester
  \node[anchor=north,draw=neutralbg,fill=white,line width=2pt,inner ysep=4pt,inner xsep=0pt,text width=\lpwidth,minimum height=1.5\lpheight,on chain,text badly centered]{#1\ LP};}

\newcommand{\question}[2]{\large\hspace{3pt}\textbf{#1}\\[3pt]\begin{minipage}{1em}~\end{minipage}\begin{minipage}{.95\linewidth}#2\end{minipage}\\[6pt]}

%%%%%%%%%%%%%%%%%%%%%%%%%%%%%%%%%%%%%%%%%%%%%%%%%%%%%%%
%%                INFOS ZUM DRUCK                    %%
%% Auf richtig konfigurierten Druckertreiber achten  %%
%%      Auf CX510: Farbkorrektur einschalten         %%
%%  Maximale Auflösung (CX510: 4800 DPI) einstellen  %%
%%      Duplex, an __kurzer__ Kante spiegeln         %%
%%%%%%%%%%%%%%%%%%%%%%%%%%%%%%%%%%%%%%%%%%%%%%%%%%%%%%%

\begin{document}
%% BLATT 1 VORDERSEITE
%% Info + Bioinfo Info-Texte
\begin{columns}
    \column[T]{\textwidth}
    \DTLsetpiesegmentcolor{1}{mathe}
\DTLsetpiesegmentcolor{2}{info}
\DTLsetpiesegmentcolor{3}{sq}
\DTLsetpiesegmentcolor{4}{andere}

\begin{Huge}
    Informatik
\end{Huge}
\begin{exampleblock}{\textcolor{white}{Was ist der Studiengang?}}
    Der grundständigste *-Informatik-Studiengang. Beinhaltet im Gegensatz zu anderen Studiengängen den meisten Umfang an technischer und theoretischer Informatik. Eine gute Portion Mathe ist außerdem dabei. Außerdem muss man 18 LP in beliebigen Veranstaltungen (überfachliche Kompetenzen übK) anderer Fächer belegen (welche nicht Sport sind). Danach kann das Studium mit einem Master (4 Semester Regelstudienzeit) weitergeführt werden.
\end{exampleblock}	

\begin{block}{Welcher Teil macht wie viel im Studium aus?}
    \begin{figure}[h!]
        \begin{minipage}{\linewidth}
            \centering
            \DTLloaddb{LPverteilung}{inhalte/info.csv}
            \tikzstyle{every node}=[text width={},minimum height=0pt]
            \DTLpiechart{
                variable=\lp,
                innerlabel={\parbox{40pt}{\centering\color{white} \bereich}},
                innerratio=0.25,
                radius=70pt,
                rotateinner}{LPverteilung}{\bereich=Bereich,\lp=LP}
        \end{minipage}
        
        \caption{Verteilung der Themenbereiche über das komplette Studium}
    \end{figure}	
\end{block}%

\begin{block}{Was macht man in welchem Semester?}
    \begin{figure}[h!]
        \vspace{-10pt}
        \resizebox{\linewidth}{!}{
        \begin{minipage}{\textwidth}
            \begin{tikzpicture}
                \begin{scope}[start chain=going below,node distance=\nodedistancecorrection]
                    \semester{1}
                    \veranstaltung{9}{Prak\-tische In\-for\-ma\-tik~1}{info}                
                    \veranstaltung{9}{Mathe\-matik f. In\-for\-ma\-tik~1}{mathe}
                    \veranstaltung{6}{Techn. In\-for\-ma\-tik~1}{info}
                    \veranstaltung{6}{übK}{sq}
                    \SumLP{30}
                \end{scope}
                \begin{scope}[xshift=1\lpwidth,start chain=going below,node distance=\nodedistancecorrection]
                    \semester{2}
                    \veranstaltung{9}{Prak\-tische In\-for\-ma\-tik~2}{info}
                    \veranstaltung{9}{Mathe\-matik f. Informatik~2}{mathe}
                    \veranstaltung{9}{Techn. Informatik~2}{info}                
                    \veranstaltung{6}{Techn. Informatik~3}{info}
                    \SumLP{33}
                \end{scope}
                \begin{scope}[xshift=2\lpwidth,start chain=going below,node distance=\nodedistancecorrection]
                    \semester{3}
                    \veranstaltung{9}{Theoretische Informatik~1}{info}
                    \veranstaltung{9}{Mathe\-matik f. Informatik~3}{mathe}
                    \veranstaltung{6}{Praktische Informatik~3}{info}
                    \veranstaltung{6}{WP Informatik}{info}
                    \SumLP{30}
                \end{scope}
                \begin{scope}[xshift=3\lpwidth,start chain=going below,node distance=\nodedistancecorrection]
                    \semester{4}                
                    \veranstaltung{9}{Theoretische In\-for\-ma\-tik~2}{info}
                    \veranstaltung{6}{Mathematik f. Informatik 4}{mathe}                
                    \veranstaltung{9}{Team\-projekt}{info}                
                    \veranstaltung{6}{Grundlagen ML}{info}
                    \SumLP{30}
                \end{scope}
                \begin{scope}[xshift=4\lpwidth,start chain=going below,node distance=\nodedistancecorrection]
                    \semester{5}
                    \veranstaltung{6}{WP Praktische Informatik}{info}
                    \veranstaltung{6}{WP Theoretische Informatik}{info}
                    \veranstaltung{6}{WP Technische Informatik}{info}
                    \veranstaltung{9}{WP Informatik}{info}
                    \veranstaltung{3}{Proseminar}{sq}
                    \SumLP{30}
                \end{scope}
                \begin{scope}[xshift=5\lpwidth,start chain=going below,node distance=\nodedistancecorrection]
                    \semester{6}
                    \veranstaltung{12}{übK}{sq}            
                    \veranstaltung{15}{Bachelor\-arbeit (inkl. Kolloquium)}{info}
                    \SumLP{27}
                \end{scope}
            \end{tikzpicture}
        \end{minipage}}
\end{figure}
 
    Das 1. Semester ist nach Plan ein Wintersemester. Wenn du dein Studium zum Sommersemester beginnen möchtest, beginnst du im Plan bei Semester 2 und machst dann Semester 1.
    Dieser Verlauf ist unabhängig vom Studienbeginn nur ein Vorschlag und kein bindender Studienplan. Es empfiehlt sich jedoch, den Plan einzuhalten, wenn man in Regelstudienzeit studieren möchte.
\end{block}

\vfill
\begin{flushright}
    \includegraphics[width=0.4\textwidth]{images/fsilogo.pdf}
\end{flushright}

\end{columns}
 \begin{columns}
    \column[T]{\textwidth}
    \colorlet{lebenswiss}{andere}
\DTLsetpiesegmentcolor{1}{info}
\DTLsetpiesegmentcolor{2}{mathe}
\DTLsetpiesegmentcolor{3}{bioinfo}
\DTLsetpiesegmentcolor{4}{lebenswiss}
\DTLsetpiesegmentcolor{5}{sq}

\begin{Huge}
    Bioinformatik
\end{Huge}

\begin{exampleblock}{\textcolor{white}{Was ist der Studiengang?}}
    Grob gesagt die Schnittstelle zwischen den Chemiker*innen im Labor und der Datenverarbeitung am Rechner. Mögliche Schwerpunkte gehen in Richtung automatisierte Verarbeitung von DNA-Daten, Drug Design, Krebsforschung etc.
    Das Studium beinhaltet neben der klassischen Informatik Inhalte aus Molekularbiologie, Neurobiologie, Biochemie und Chemie. Danach kann das Studium mit einem Master (4 Semester Regelstudienzeit) weitergeführt werden.
\end{exampleblock}

\begin{block}{Welcher Teil macht wie viel im Studium aus?}
    \begin{figure}[h!]
        \begin{minipage}{\linewidth}
            \centering
            \DTLloaddb{LPverteilungBio}{inhalte/bioinfo.csv}
            \tikzstyle{every node}=[text width={},minimum height=0pt]
            \DTLpiechart{
                variable=\lp,
                innerlabel={\parbox{40pt}{\centering\color{white} \bereich}},
                innerratio=0.25,
                radius=70pt,
                rotateinner}{LPverteilungBio}{\bereich=Bereich,\lp=LP}
        \end{minipage}
        \vspace{-20pt}
	\caption{Verteilung der Themenbereiche über das komplette Studium}
    \end{figure}
\end{block}

\begin{block}{Was macht man in welchem Semester?}
    \vspace{-10pt}
    \begin{figure}[h!]
        \resizebox{\linewidth}{!}{
        \begin{minipage}{\textwidth}
            \begin{tikzpicture}
                \begin{scope}[start chain=going below,node distance=\nodedistancecorrection]
                    \semester{1}
                    \veranstaltung{9}{Praktische In\-for\-ma\-tik~1}{info}
                    \veranstaltung{9}{Mathe\-matik f. Informatik~1}{mathe}
                    \veranstaltung{6}{Biomoleküle und Zelle}{lebenswiss}
                    \veranstaltung{9}{Anorg. und Org. Chemie}{lebenswiss}
                    \SumLP{33}
                \end{scope}
                \begin{scope}[xshift=1\lpwidth,start chain=going below,node distance=\nodedistancecorrection]
                    \semester{2}
                    \veranstaltung{9}{Praktische In\-for\-ma\-tik~2}{info}
                    \veranstaltung{9}{Mathe\-matik~f. Informatik~2}{mathe}
                    \veranstaltung{3}{Einf. Bioinformatik}{bioinfo}
                    \veranstaltung{3}{Biochemie}{lebenswiss}
                    \veranstaltung{3}{übK}{sq}
                    \SumLP{27}
                \end{scope}
                \begin{scope}[xshift=2\lpwidth,start chain=going below,node distance=\nodedistancecorrection]
                    \semester{3}
                    \veranstaltung{9}{Theoretische Informatik 1}{info}
                    \veranstaltung{6}{Praktische Informatik 3}{info}
                    \veranstaltung{9}{Mathe\-matik~f. Informatik 3}{mathe}
                    \veranstaltung{9}{Neurobiologie}{lebenswiss}
                    \SumLP{33}
                \end{scope}
                \begin{scope}[xshift=3\lpwidth,start chain=going below,node distance=\nodedistancecorrection]
                    \semester{4}
                    \veranstaltung{9}{The\-o\-re\-ti\-sche Informatik 2}{info}
                    \veranstaltung{6}{Stochastik}{mathe}
                    \veranstaltung{9}{Grundlagen der Bioinformatik}{bioinfo}
                    \veranstaltung{9}{übK: Teamprojekt}{sq}
                    \SumLP{33}
                \end{scope}
                \begin{scope}[xshift=4\lpwidth,start chain=going below,node distance=\nodedistancecorrection]
                    \semester{5}
                    \veranstaltung{6}{Wahlpflicht Informatik}{info}
                    \veranstaltung{6}{Physikalische Chemie}{lebenswiss}
                    \veranstaltung{6}{Molekular-Biologie}{lebenswiss}
                    \veranstaltung{6}{WP Lebenswissen\-schaf.}{lebenswiss}
                    \veranstaltung{3}{übK: Proseminar}{sq}
                    \SumLP{27}
                \end{scope}
                \begin{scope}[xshift=5\lpwidth,start chain=going below,node distance=\nodedistancecorrection]
                    \semester{6}
                    \veranstaltung{6}{WP Bioinformatik}{bioinfo}
                    \veranstaltung{6}{übK}{sq}
                    \veranstaltung{15}{Bachelor\-arbeit}{bioinfo}
                    \SumLP{27}
                \end{scope}
            \end{tikzpicture}
        \end{minipage}}
    \end{figure}

    Das 1. Semester ist nach Plan ein Wintersemester. Wenn du dein Studium zum Sommersemester beginnen möchtest, beginnst du im Plan bei Semester 2 und machst dann Semester 1. 
    Dieser Verlauf ist unabhängig vom Studienbeginn nur ein Vorschlag und kein bindender Studienplan. Es empfiehlt sich jedoch, den Plan einzuhalten, wenn man in Regelstudienzeit studieren möchte.
\end{block}

\vfill
\begin{flushright}
    \includegraphics[width=0.4\textwidth]{images/fsilogo.pdf}
\end{flushright}

\end{columns}
\newpage 
%% BLATT 2 VORDERSEITE
%% Medien- und Medizininfo Infotexte
\begin{columns}
    \column[T]{\textwidth}
    \colorlet{medienwiss}{andere}
\DTLsetpiesegmentcolor{1}{info}
\DTLsetpiesegmentcolor{2}{mathe}
\DTLsetpiesegmentcolor{3}{medieninfo}
\DTLsetpiesegmentcolor{4}{medienwiss}
\DTLsetpiesegmentcolor{5}{sq}

\begin{Huge}
    Medieninformatik
\end{Huge}

\begin{exampleblock}{\textcolor{white}{Was ist der Studiengang?}}
    Irgendwas mit Medien? Nicht so wirklich. Die Medieninformatik beschäftigt sich mit User Interfaces, Nutzerinteraktion, modernen Techniken wie Eye Tracking, macht Ausflüge in die Medienwissenschaft aber ist auch zu großen Teilen Informatik- und Mathematik-lastig. Danach kann das Studium mit einem Master (4 Semester Regelstudienzeit) weitergeführt werden.
\end{exampleblock}

\begin{block}{Welcher Teil macht wie viel im Studium aus?}
    \begin{figure}[h!]
        \vspace{-10pt}
        \begin{minipage}{\linewidth}
            \centering
            \DTLloaddb{LPverteilungMedien}{inhalte/medieninfo.csv}
            \tikzstyle{every node}=[text width={},minimum height=0pt]
            \DTLpiechart{
                variable=\lp,
                innerlabel={\parbox{40pt}{\centering\color{white} \bereich}},
                innerratio=0.25,
                radius=70pt,
                rotateinner}{LPverteilungMedien}{\bereich=Bereich,\lp=LP}
        \end{minipage}
        \caption{Verteilung der Themenbereiche über das komplette Studium}
    \end{figure}
\end{block}

\begin{block}{Was macht man in welchem Semester?}
    \vspace{-10pt}
    \begin{figure}[h!]
        \resizebox{\linewidth}{!}{
        \begin{minipage}{\textwidth}
            \begin{tikzpicture}
                \begin{scope}[start chain=going below,node distance=\nodedistancecorrection]
                    \semester{1}
                    \veranstaltung{9}{Praktische In\-for\-ma\-tik~1}{info}
                    \veranstaltung{9}{Mathe\-matik f. Informatik~1}{mathe}
                    \veranstaltung{6}{User-Experience}{medieninfo}
                    \veranstaltung{9}{WP Medienwiss.}{medienwiss}
                    \SumLP{33}
                \end{scope}
                \begin{scope}[xshift=1\lpwidth,start chain=going below,node distance=\nodedistancecorrection]
                    \semester{2}
                    \veranstaltung{9}{Praktische In\-for\-ma\-tik~2}{info}
                    \veranstaltung{9}{Technische Informatik~2}{info}
                    \veranstaltung{9}{Mathe\-matik f. Informatik~2}{mathe}
                    \veranstaltung{6}{Einführung in die Internettech.}{medieninfo}
                    \SumLP{33}
                \end{scope}
                \begin{scope}[xshift=2\lpwidth,start chain=going below,node distance=\nodedistancecorrection]
                    \semester{3}
                    \veranstaltung{9}{Theoretische Informatik 1}{info}
                    \veranstaltung{6}{Praktische Informatik 3}{info}
                    \veranstaltung{9}{Mathe\-matik f. Informatik~3}{mathe}
                    \veranstaltung{6}{Grundlagen der Multimediatechnik}{medieninfo}
                    \SumLP{30}
                \end{scope}
                \begin{scope}[xshift=3\lpwidth,start chain=going below,node distance=\nodedistancecorrection]
                    \semester{4}                
                    \veranstaltung{9}{The\-o\-re\-ti\-sche Informatik~2}{info}
                    \veranstaltung{9}{Team\-projekt}{sq}
                    \veranstaltung{6}{WP Informatik}{info}
                    \veranstaltung{3}{Ethik-Proseminar}{sq} 
                    \SumLP{30}
                \end{scope}
                \begin{scope}[xshift=4\lpwidth,start chain=going below,node distance=\nodedistancecorrection]
                    \semester{5}
                    \veranstaltung{6}{Bildverarbeitung}{medieninfo}
                    \veranstaltung{9}{Graphische Datenverarbeitung}{medieninfo}
                    \veranstaltung{6}{WP Medieninformatik}{medieninfo}
                    \veranstaltung{3}{WP Medienwiss.}{medienwiss}
                    \veranstaltung{3}{Proseminar}{sq}
                    \SumLP{27}
                \end{scope}
                \begin{scope}[xshift=5\lpwidth,start chain=going below,node distance=\nodedistancecorrection]
                    \semester{6}
                    \veranstaltung{9}{WP Medieninformatik}{medieninfo}
                    \veranstaltung{6}{übK}{sq}
                    \veranstaltung{15}{Bachelor\-arbeit inkl Vortrag}{medieninfo}
                    \SumLP{27}
                \end{scope}
            \end{tikzpicture}
        \end{minipage}}  
    \end{figure}
    
    Das 1. Semester ist nach Plan ein Wintersemester, der Studienbeginn ist hier auch nur zum Wintersemester möglich. 
    Dieser Verlauf ist lediglich ein Vorschlag und kein bindender Studienplan. Es empfiehlt sich jedoch, den Plan einzuhalten, wenn man in Regelstudienzeit studieren möchte.
\end{block}

\vfill
\begin{flushright}
    \includegraphics[width=0.4\textwidth]{images/fsilogo.pdf}
\end{flushright}

\end{columns}
\begin{columns}
    \column[T]{\textwidth}
    \colorlet{biologie}{andere}
\colorlet{physik}{andere}
\DTLsetpiesegmentcolor{1}{info}
\DTLsetpiesegmentcolor{2}{mathe}
\DTLsetpiesegmentcolor{3}{medizininfo}
\DTLsetpiesegmentcolor{4}{andere}
\DTLsetpiesegmentcolor{5}{sq}

\begin{Huge}
    Medizininformatik
\end{Huge}

\begin{exampleblock}{\textcolor{white}{Was ist der Studiengang?}}
    Die Schnittstelle zwischen Klinikum, Ärzt*innen und Medizintechniker*innen. Klassische Anwendungsbereiche sind E-Health, Medizinische Datenverarbeitung sowie die (Mit-)Entwicklung von Medizingeräten. Es wird ein stärkerer Fokus \\  auf Biologie, Physik und medizinische Inhalte gelegt. Danach kann das Studium mit einem Master (4 Semester Regelstudienzeit) weitergeführt werden.
\end{exampleblock}

\begin{block}{Welcher Teil macht wie viel im Studium aus?}
    \begin{figure}[h!]
        \vspace{-20pt}
        \begin{minipage}{\linewidth}
            \centering
            \DTLloaddb{LPverteilungMedizin}{inhalte/medizininfo.csv}
            \tikzstyle{every node}=[text width={},minimum height=0pt]
            \DTLpiechart{
                variable=\lp,
                innerlabel={\parbox{40pt}{\centering\color{white} \bereich}},
                innerratio=0.25,
                radius=70pt,
                rotateinner}{LPverteilungMedizin}{\bereich=Bereich,\lp=LP}
        \end{minipage}
        \vspace{-20pt}
        \caption{Verteilung der Themenbereiche über das komplette Studium}
    \end{figure}
\end{block}

\begin{block}{Was macht man in welchem Semester?}
    \vspace{-10pt}
    \begin{figure}[h!]
        \resizebox{\linewidth}{!}{
        \begin{minipage}{\textwidth}
            \begin{tikzpicture}
                \begin{scope}[start chain=going below,node distance=\nodedistancecorrection]
                    \semester{1}
                    \veranstaltung{9}{Praktische In\-for\-ma\-tik~1}{info}
                    \veranstaltung{9}{Mathe\-matik f. Informatik~1}{mathe}
                    \veranstaltung{6}{Grundlagen der Medizininformatik}{medizininfo}
                    \veranstaltung{3}{Humanbiologie I}{biologie}
                    \veranstaltung{3}{Med. Terminologie}{biologie}
                    \SumLP{30}
                \end{scope}
                \begin{scope}[xshift=1\lpwidth,start chain=going below,node distance=\nodedistancecorrection]
                    \semester{2}
                    \veranstaltung{9}{Praktische In\-for\-ma\-tik~2}{info}
                    \veranstaltung{6}{Einf. Internet\-technologien}{info}
                    \veranstaltung{9}{Mathe\-matik f. Informatik~2}{mathe}
                    \veranstaltung{6}{Humanbiologie II}{biologie}
                    \SumLP{30}
                \end{scope}
                \begin{scope}[xshift=2\lpwidth,start chain=going below,node distance=\nodedistancecorrection]
                    \semester{3}
                    \veranstaltung{6}{User Experience}{info}
                    \veranstaltung{6}{Praktische Informatik 3}{info}
                    \veranstaltung{6}{Physik I}{physik}
                    \veranstaltung{6}{Humanbiologie III}{biologie}                
                    \veranstaltung{3}{Biostatistik}{biologie}                
                    \veranstaltung{3}{Ethik (übK)}{sq}
                    \SumLP{30}
                \end{scope}
                \begin{scope}[xshift=3\lpwidth,start chain=going below,node distance=\nodedistancecorrection]
                    \semester{4}                
                    \veranstaltung{9}{Grundlagen Bioinformatik}{medizininfo}
                    \veranstaltung{6}{Physik II}{physik}                
                    \veranstaltung{6}{Humanbiologie IV}{biologie}
                    \veranstaltung{9}{Team\-projekt}{sq}                
                    \SumLP{30}
                \end{scope}
                \begin{scope}[xshift=4\lpwidth,start chain=going below,node distance=\nodedistancecorrection]
                    \semester{5}
                    \veranstaltung{6}{WP Informatik}{info}
                    \veranstaltung{6}{Medizinische Visualisierung}{medizininfo}
                    \veranstaltung{6}{Telemedizin}{medizininfo}
                    \veranstaltung{6}{WP Medizin/Biologie}{biologie}
                    \veranstaltung{3}{Proseminar}{sq}
                    \veranstaltung{3}{übK}{sq}
                    \SumLP{30}
                \end{scope}
                \begin{scope}[xshift=5\lpwidth,start chain=going below,node distance=\nodedistancecorrection]
                    \semester{6}
                    \veranstaltung{6}{WP Bioinformatik}{medizininfo}
                    \veranstaltung{6}{WP Medizininformatik}{medizininfo}
                    \veranstaltung{15}{Bachelor\-arbeit}{medizininfo}
                    \veranstaltung{3}{übK}{sq}
                    \SumLP{30}
                \end{scope}
            \end{tikzpicture}
        \end{minipage}}
    \end{figure}
    
    Das 1. Semester ist nach Plan ein Wintersemester, der Studienbeginn ist hier auch nur zum Wintersemester möglich. 
    Dieser Verlauf ist lediglich ein Vorschlag und kein bindender Studienplan. Es empfiehlt sich jedoch, den Plan einzuhalten, wenn man in Regelstudienzeit studieren möchte.
\end{block}

\vfill
\begin{flushright}
    \includegraphics[width=0.4\textwidth]{images/fsilogo.pdf}
\end{flushright}

\end{columns}
\newpage 
%% BLATT 3 VORDERSEITE
%% Kogni und Lehramt Infotexte
\begin{columns}
    \column[T]{\textwidth}
    \colorlet{other}{brown}
\colorlet{philo}{other}
\colorlet{lingu}{other}
\colorlet{neuro}{other}
\colorlet{psycho}{other}
\colorlet{bscarb}{other}


\DTLsetpiesegmentcolor{1}{kogni}
\DTLsetpiesegmentcolor{2}{info}
\DTLsetpiesegmentcolor{3}{mathe}
\DTLsetpiesegmentcolor{4}{other}
\DTLsetpiesegmentcolor{5}{sq}

\begin{Huge}
    Kognitionswissenschaft
\end{Huge}

\begin{exampleblock}{\textcolor{white}{Was ist der Studiengang?}}
    Ein sehr interdisziplinärer Studiengang, der einzelne Aspekte der Informatik, (Neuro-)Biologie, Linguistik, Philosophie und Psychologie miteinander verbindet, bzw. auch einzeln behandelt. Alle Fragen, die dem Denken gewidmet sind, finden hier ihren Platz und werden mithilfe der verschiedenen Sichtweisen der unterschiedlichen Disziplinen versucht zu beantworten. Danach kann das Studium mit einem Master (4 Semester Regelstudienzeit) weitergeführt werden.
\end{exampleblock}

\begin{block}{Welcher Teil macht wie viel im Studium aus?}
    \begin{figure}[h!]
        \vspace{-10pt}
        \begin{minipage}{\linewidth}
            \centering
            \DTLloaddb{LPverteilungKogni}{inhalte/kogni.csv}
            \tikzstyle{every node}=[text width={},minimum height=0pt]
            \DTLpiechart{
                variable=\lp,
                innerlabel={\parbox{40pt}{\centering\color{white} \bereich}},
                innerratio=0.25,
                radius=70pt,
                rotateinner}{LPverteilungKogni}{\bereich=Bereich,\lp=LP}
        \end{minipage}
        \vspace{-20pt}
        \caption{Verteilung der Themenbereiche über das komplette Studium}
    \end{figure}
\end{block}

\begin{block}{Was macht man in welchem Semester?}
    \vspace{-10pt}
    \begin{figure}[h!]
        \resizebox{\linewidth}{!}{
        \begin{minipage}{\textwidth}
            \begin{tikzpicture}
                \begin{scope}[start chain=going below,node distance=\nodedistancecorrection]
                    \semester{1}
                    \veranstaltung{9}{Prak\-tische In\-for\-ma\-tik~I}{info}
                    \veranstaltung{9}{Mathe\-matik f. Informatik~I}{mathe}        
                    %\veranstaltung{6}{Neuro\-bio\-logie und Sinnes\-physio\-logie}{neuro}
                    \veranstaltung{3}{Forschungsm. I}{kogni}
                    %\veranstaltung{3}{Einf.\ Kognition}{kogni}
                    \veranstaltung{3}{Comp. Statistik I}{kogni}
                    \veranstaltung{6}{Grundlagen\ der \ KogWis}{kogni}
                    \SumLP{30}
                \end{scope}
                \begin{scope}[xshift=1\lpwidth,start chain=going below,node distance=\nodedistancecorrection]
                    \semester{2}
                    \veranstaltung{9}{Prak\-tische In\-for\-ma\-tik~2}{info}
                    \veranstaltung{9}{Mathe\-matik f. Informatik~II}{mathe}
                    \veranstaltung{3}{Allg. Psych. C/D}{psycho}
                    \veranstaltung{3}{Comp. Statistik II}{kogni}
                    \veranstaltung{3}{Forschungsm. II}{kogni}
                    \SumLP{27}
                \end{scope}
                \begin{scope}[xshift=2\lpwidth,start chain=going below,node distance=\nodedistancecorrection]
                    \semester{3}
                    \veranstaltung{9}{Theoretische Informatik~I}{info}
                    \veranstaltung{9}{Mathe\-matik f. Informatik~III}{mathe}
                    \veranstaltung{6}{Linguistik}{lingu}
                    \veranstaltung{6}{Exp. Kogwiss.}{kogni}
                    \veranstaltung{3}{Allg. Psych. B}{psycho}
                    \SumLP{33}
                \end{scope}
                \begin{scope}[xshift=3\lpwidth,start chain=going below,node distance=\nodedistancecorrection]
                    \semester{4}     
                    \veranstaltung{9}{WP Team\-projekt}{sq}
                    \veranstaltung{6}{Philosophie}{philo}
                    \veranstaltung{6}{Language \& Cogn.}{lingu}
                    \veranstaltung{6}{Kog. Architekturen}{kogni}
                    \veranstaltung{3}{Forschungsseminar}{psycho}
                    \SumLP{30}
                \end{scope}
                \begin{scope}[xshift=4\lpwidth,start chain=going below,node distance=\nodedistancecorrection]
                    \semester{5}
                    \veranstaltung{6}{Kognitions\-informatik}{info}
                    \veranstaltung{6}{Computational Neuroscience}{neuro}
                    \veranstaltung{12}{Vertiefung Kogwiss.}{kogni}
                    \veranstaltung{3}{Forschungsseminar}{kogni}
                    \veranstaltung{6}{Perception}{psycho}
                    \SumLP{33}
                \end{scope}
                \begin{scope}[xshift=5\lpwidth,start chain=going below,node distance=\nodedistancecorrection]
                    \semester{6}
                    \veranstaltung{15}{Bachelor\-arbeit}{kogni}
                    \veranstaltung{9}{übK}{sq}
                    %\veranstaltung{3}{Forschungsseminar}{kogni}
                    \veranstaltung{3}{Forschungskoll.}{sq}
                    \SumLP{27}
                \end{scope}
            \end{tikzpicture}
        \end{minipage}}        
    \end{figure}

    Das 1. Semester ist nach Plan ein Wintersemester, der Studienbeginn ist hier auch nur zum Wintersemester möglich. 
    Dieser Verlauf ist lediglich ein Vorschlag und kein bindender Studienplan. Es empfiehlt sich jedoch, den Plan einzuhalten, wenn man in Regelstudienzeit studieren möchte.
\end{block}

\vfill
\begin{flushright}
    \includegraphics[width=0.4\textwidth]{images/fsilogo.pdf}
\end{flushright}

\end{columns}
\begin{columns}
    \column[T]{\textwidth}
    \input{poster/poster_lehramt.tex}
\end{columns}
\newpage 
\end{document}
